% --------------------------------------------------------------
% This is all preamble stuff that you don't have to worry about.
% Head down to where it says "Start here"
% --------------------------------------------------------------
 
\documentclass[12pt]{article}
 
\usepackage[margin=1in]{geometry} 
\usepackage{amsmath,amsthm,amssymb}

\newtheorem{case}{Case}
 
\newcommand{\N}{\mathbb{N}}
\newcommand{\Z}{\mathbb{Z}}
\newcommand{\Q}{\mathbb{Q}}
\newcommand{\R}{\mathbb{R}}
\newcommand{\tr}{\textnormal{tr}}
\newcommand{\iCup}{$\small$\bigcup\limits_{i \in I}$\normalsize$}
\newcommand{\iCap}{\bigcap\limits_{i \in I}}
\newcommand{\diam}{\textnormal{diam}}
\newcommand{\norm}[1]{\left\lVert#1\right\rVert}
\newcommand{\dist}{\textnormal{dist}}


 
\newenvironment{theorem}[2][Theorem]{\begin{trivlist}
\item[\hskip \labelsep {\bfseries #1}\hskip \labelsep {\bfseries #2.}]}{\end{trivlist}}
\newenvironment{lemma}[2][Lemma]{\begin{trivlist}
\item[\hskip \labelsep {\bfseries #1}\hskip \labelsep {\bfseries #2.}]}{\end{trivlist}}
\newenvironment{exercise}[2][Exercise]{\begin{trivlist}
\item[\hskip \labelsep {\bfseries #1}\hskip \labelsep {\bfseries #2.}]}{\end{trivlist}}
\newenvironment{problem}[2][Problem]{\begin{trivlist}
\item[\hskip \labelsep {\bfseries #1}\hskip \labelsep {\bfseries #2.}]}{\end{trivlist}}
\newenvironment{question}[2][Question]{\begin{trivlist}
\item[\hskip \labelsep {\bfseries #1}\hskip \labelsep {\bfseries #2.}]}{\end{trivlist}}
\newenvironment{corollary}[2][Corollary]{\begin{trivlist}
\item[\hskip \labelsep {\bfseries #1}\hskip \labelsep {\bfseries #2.}]}{\end{trivlist}}
 
\begin{document}
 
% --------------------------------------------------------------
%                         Start here
% --------------------------------------------------------------
 
\title{Weekly Homework 10}
\author{George Duncan MATH 4317} 
\maketitle
\date


\begin{problem}\label{1)}
	$\Sigma = $
	$\begin{bmatrix}
	1&r\\r&1
	\end{bmatrix}$
	\\
	We must have $\Sigma$ positive definite. By the Cauchy-Schwartz inequality:\tabularnewline\\
	$\textnormal{cov}(X_1, X_2)^2 \leq \textnormal{cov}(X_1, X_1) \textnormal{cov}(X_2, X_2)$\\
	$\textnormal{cov}(X_1, X_2)^2 \leq \textnormal{var}(X_1) \textnormal{var}(X_2)$\\
	$r^2 \leq 1$ ... $r\in[-1, 1]$\\
	For both $r = -1, r = 1$, the determinant of the matrix is 0, therefore the matrix is degenerate. Thus: $r \in (-1, 1)$. We can use the following expansion for the joint probability density function:\\
	$$\Pr(X = x) = p(X) = \frac{1}{2\pi}\det(\Sigma)^{-1}\exp(-q(X-\mu))$$, where $q$ is defined by the following:\\
	$$q(Z) = \frac{\frac{z_1^2}{\Sigma_{1,1}^2} - \frac{2\sqrt{1-\det(\Sigma)}z_1z_2}{\tr(\Sigma)} + \frac{z_2^2}{\Sigma_{2,2}^2}}{2\det(\Sigma)}$$
	Substituting in the values form this problem:\\
	$$p(X) = \frac{1}{2\pi(1-r^2)}\exp(-\frac{(x_1-\mu_1)^2 - 2r(x_1-\mu_1)(x_2-\mu_2) + (x_2-\mu_2)^2}{2(1-r^2)})$$
	\\
	\\
	\\
	For the eigen values we must set the determinant of the matrix equal to 0:\\
	$\det(\Sigma) = 0 = (1-\lambda)^2-r^2$, thus $\lambda = 1 \pm r$. Consider the positive case $\lambda_1$, and the negative case $\lambda_2$. For each $\lambda$ we can find the corresponding eigen vectors accordingly:\\
	$(\Sigma - \lambda I)v = 0$\\
	$\begin{bmatrix}
	-r&r\\r&-r\\
	\end{bmatrix}v_1 = $
	$\begin{bmatrix}0\\0 \end{bmatrix}$, thus $v_1 = c(1, 1) : c\in \R$, after normalization:
	$v_1 = (\frac{\sqrt(2)}{2}, \frac{\sqrt(2)}{2})$ 
	$\begin{bmatrix}
	r&r\\r&r\\
	\end{bmatrix}v_1 = $
	$\begin{bmatrix}0\\0 \end{bmatrix}$, thus $v_2 = c(1, -1) : c\in \R$, after normalization:
	$v_2 = (\frac{\sqrt(2)}{2}, \frac{-\sqrt(2)}{2})$ 
\end{problem}
\end{document}